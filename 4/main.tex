\documentclass[12pt]{article}
\usepackage[margin=1in]{geometry} 
\usepackage[shortlabels]{enumitem}
\usepackage{caption}
\usepackage{algorithm}
\usepackage[noend]{algpseudocode}
\usepackage[table,xcdraw]{xcolor}
\usepackage{import}
\usepackage{tikz}
\usepackage{tikz,fullpage}
\usetikzlibrary{arrows, automata}
\usepackage{tkz-berge}
\usepackage[position=top]{subfig}
\usepackage{amsmath,amsthm,amssymb,amsfonts, enumitem, fancyhdr, color, comment, graphicx, environ}
\pagestyle{fancy}
\setlength{\headheight}{65pt}
\newenvironment{problem}[2][Problem]{\begin{trivlist}
\item[\hskip \labelsep {\bfseries #1}\hskip \labelsep {\bfseries #2.}]}{\end{trivlist}}
\newenvironment{sol}
    {\emph{Solution:}
    }
    {
    \qed
    }
\specialcomment{com}{ \color{blue} \textbf{Comment:} }{\color{black}} %for instructor comments while grading
\NewEnviron{probscore}{\marginpar{ \color{blue} \tiny Problem Score: \BODY \color{black} }}
% creates keywords for input and output in algorithm block
\algblock{Input}{EndInput}
\algnotext{EndInput}
\algblock{Output}{EndOutput}
\algnotext{EndOutput}
\newcommand{\Desc}[2]{\State \makebox[13em][l]{#1}#2}
\lhead{Mazen Alotaibi \textit{(alotaima)}}
\rhead{CS 420 \\ Section 001 \\ Winter 2019 \\ HW 4}

\begin{document}

\begin{problem}{1: Ford-Fulkerson.} In this problem, we will... 
\begin{enumerate}[(a)]
    \item Consider the network $G$ shown in Fig. 1 a)...
    \item Find any augmenting path in $G_f$...
    \item Show the residual network...
    \item Is this the final maximum flow in this...
    \item Show the residual network for your...
    \item Show the cut that results by partitioning...
\end{enumerate}
\end{problem}
\begin{sol}
\begin{enumerate}[(a)]
    \item Please look at Graph 1 for the residual network $G_f$.
    \item Please look at Graph 1. The path $s->d->b->a->t$ has the maximum flow to push is 2.
    \item Please look at Graph 2 for the updated flow. Please look at Graph 4 for the updated residual network $G_{f'}$.
    \item There isn't any path from $s$ to $t$. The maximum flow is 13.
    \item The residual graph is the same as (c) because the flow from (c) was already the maximum.
    \item Please look at Graph 5 for reference to my answer. The minimum cuts are $(\{s, b, c, d\},\{a, t\})$ and $(\{s, a, b, c, d\},\{t\})$, and the capacity of both of the cuts is equal to the sum of capacities from X region to Y region in both of the suggested minimum cuts, which they have a capacity of 13.
\end{enumerate}
\end{sol}
\begin{problem}{2: Disjoint Roads.} A number k of trucking companies, $c_1$, . . . , $c_k$...
\end{problem}
\begin{sol}
This problem is a max flow problem. By using Ford-Fulkerson algorithm, the result of the algorithm is a set of paths. If the value of the flow is less than k, the algorithm returns false, impossible. The time complexity is $O(Ek)$ as the max flow $|f|<k$.
\end{sol}
\begin{lstlisting}[language=Python]
def Distance(a,b):
    return int(round(math.sqrt((math.pow(a['i'] - b['i'],2))+(math.pow(a['j'] - b['j'],2)))))

def KNN(cities, inFile):
    matrix = [[-1 for x in range(len(cities))] for y in range(len(cities))]
    minLength = sys.maxsize
    order = []
    for x in range(len(cities)):
        allCities = [z for z in cities]
        route = []
        route.append(allCities[x]['city'])
        allCities.remove(allCities[x])
        length = 0
        while len(allCities) > 0:
            current = cities[route[len(route)-1]]
            minDistance = sys.maxsize
            minCity = -1
            for y in range(len(allCities)):
                currentDistance = matrix[current['city']][allCities[y]['city']]
                if currentDistance == -1:
                    currentDistance = Distance(current, allCities[y])
                    matrix[current['city']][allCities[y]['city']] = currentDistance
                    matrix[allCities[y]['city']][current['city']] = currentDistance
                if currentDistance < minDistance:
                    minDistance = currentDistance
                    minCity = allCities[y]
            route.append(minCity['city'])
            allCities.remove(minCity)
            length += minDistance
        currentDistance = matrix[cities[route[0]]['city']][cities[route[len(route)-1]]['city']]
        if currentDistance == -1:
            currentDistance = Distance(cities[route[0]], cities[route[len(route)-1]])
            matrix[cities[route[0]]['city']][cities[route[len(route)-1]]['city']] = currentDistance
            matrix[cities[route[len(route)-1]]['city']][cities[route[0]]['city']] = currentDistance
        length += currentDistance
        if length < minLength:
            minLength = length
            order = [x for x in route]
            WriteFileKNN(inFile, order, minLength)
\end{lstlisting}
\newpage

\begin{figure}[h]
\begin {center}
\includegraphics[width=\textwidth]{fig/1.jpg}
\caption*{Graph 1: the residual network $G_f$}
\end{center}
\centering
\end{figure}

\begin{figure}[h]
\begin {center}
\includegraphics[width=\textwidth]{fig/2.jpg}
\caption*{Graph 2: the updated flow}
\end{center}
\centering
\end{figure}

\begin{figure}[h]
\begin {center}
\includegraphics[width=\textwidth]{fig/3.jpg}
\caption*{Graph 3: the updated residual network}
\end{center}
\centering
\end{figure}

\begin{figure}[h]
\begin {center}
\includegraphics[width=\textwidth]{fig/4.jpg}
\caption*{Graph 4: the graph before applying the minimum cut. \textit{Note: $s->d$ is 7 and $a->t$ is 6}}
\end{center}
\centering
\end{figure}

\begin{figure}[h]
\begin {center}
\includegraphics[width=\textwidth]{fig/5.jpg}
\caption*{Graph 5: the minimum cut for two cuts. \textit{Note: $b->a$ is 2 and $b->c$ is 3}

The purple line is for the $(\{s, b, c, d\},\{a, t\})$ maximum cut.

The black line for the $(\{s, a, b, c, d\},\{t\})$ maximum cut.}
\end{center}
\centering
\end{figure}


\end{document}
